
\chapter*{Abstract}


% \AddToHook{shipout/background}{%
%     \put (0in,-\paperheight){\includegraphics[width=\paperwidth,height=\paperheight]{images/Preprint_bg5.jpg}}%
% }

Drone development is an evolving field of research. Tasks are performed by drones and this can be simplified by the use of several technologies. In this thesis, I examine how smart systems benefit UAVs, and the tasks in which they operate. The problem I address is to understand the tradeoffs of certain smart systems over others in assisting UAV functionality.

The first approach explores the potential of distributed systems to assist task development upon UAVs. We document the design of a development and demonstration testbed conform to existing research. We describe a procedural task-based architecture to complement an existing swarm stack. We demonstrate that the runtime environment is capable of coordinating multiple robots. A custom high level interface wraps the testbed for more complex tasks, and this is demonstrated in a multi-drone choreography. There are evident tradeoffs for a Flight Arena in terms of available space, constraining the range of tasks. This is however redeemed by a greater reliability of demonstrations due to the assistance of monitoring tools, and other forms of choreographic automation.

We investigate a Mixed Reality Interface for the Testbed, as well as methods of drone Piloting using a Computer Vision algorithm. The utility of the framework is demonstrated with two different tasks: quadrotor piloting using computer vision and collision-free flight of multiple UAVs. Building on existing frameworks like MediaPipe Hands, and Unity3D, we create perception pipelines for semi-autonomous flight, and we proceed to evaluate the response latency of these pipelines. The testbed of the first chapter has shown its benefits as we develop two Human-Drone Experiments. By managing the distribution of platform resources, the distributed system provides an abstraction whereas the development focused on the performance of several tasks.

The second approach explores the potential of  onboard systems to assist the real-world deployment of UAVs. Applications are explored for UAVs as Mobile Sensing Platforms, with high-sampling and high-precision equipment. We design a carrier drone with Onboard Data Acquisition and we put it to practice along standards defined by industrial practitioners. Two payloads are tested in outdoor flight, for atmospheric data and vibration data. We characterise the sensors used on these payloads. A vibration probe is designed and our tests demonstrate its relevance in the field of mobile sensing. The use of a carrier drone has shown to be very beneficial, as it could carry high-frequency, high-sensitivity data acquisition equipment that were advantageous in the field of remote sensing: from detecting footsteps, to sudden changes in luminosity, to assisting with the optimal damping of the drone, to characterising an accelerometer mount, onboard data collection is an interesting testing ground for innovation.




\chapter*{Résumé}

% \textbf{Le domaine des UAVs} se développe rapidement, et beaucoup envisagent que les tâches de service seront un jour facilitées par les UAV. Cette thèse est dédiée à démontrer que les \textbf{tâches de service} à base de drones sont désormais possibles, que ce soit en intérieur ou sur un terrain ouvert. Il le fait en reliant les approches de recherche pour la robotique UAV aux approches industrielles. Tout d'abord, une \textbf{arene de vol} est développée conformément aux recherches existantes. Ensuite, l'utilité du framework est démontrée grace a deux tâches différentes : \textbf{le pilotage de quadrotor a l'aide de computer vision} et \textbf{collision avoidance} de plusieurs drones. En nous appuyant sur des frameworks existants comme MediaPipe Hands et Unity3D, nous developpons des \textbf{pipelines de perception} pour un vol semi-autonome. Une \textbf{interface de gestion des essaims}, la Swarm Programming Interface, est intégrée dans la configuration de l'essaim pour une utilisation par des recherches ultérieures. Enfin, des applications sont explorées pour les drones en tant que plateformes de capteurs mobiles, avec \textbf{équipements à échantillonnage élevé et de haute précision}. Une \textbf{sonde vibratoire} est conçue et nos tests démontrent sa pertinence dans le domaine des drones de service. Cet axe de travail intéresse les praticiens du drone car nos partenaires industriels ont sollicité une enveloppe Soleau sur cette solution.

Dans cette thèse, j'examine comment le domaine des drones peut profiter de systèmes intelligents pour effectuer des tâches. En comparant ces systèmes intelligents, ma problématique adresse les compromis de ces systèmes dans l'assistance aux fonctionnalités UAV. 

La première approche explore le potentiel des systèmes distribués pour permettre le développement de tâches complexes de drones. Nous documentons la conception d'un banc d'essai pour le développement et de la démonstration conforme à d'autres laboratoires de drones.Nous mettons en place une architecture procédurale fondee sur l'execution de tâches. Nous démontrons que cet environnement d'exécution est capable de coordonner plusieurs robots. Une interface personnalisée de haut niveau enveloppe le banc d'essai pour effectuer des tâches plus complexes et son utilisation permet une chorégraphie multi-drone. Pour revenir à la problématique, il existe des des contraintes d'espace pour une arène de vol, ce qui peut limiter les possibilites de tâches. Cela est compensé par l'assistance d'outils de supervision et d'automatisation qui permettent une meilleure fiabilité des démonstrations. 

Nous mettons en place une interface de réalité mixte au sein du banc d'essai, ainsi qu'une methode de pilotage de drones à l'aide d'un algorithme de computer vision. Ces deux tâches mettent en relief le type de taches qui peuvent etre developpees : le pilotage de quadrotor avec le computer vision et le vol anti-collision de plusieurs drones. En nous appuyant sur des cadres existants tels que MediaPipe Hands et Unity3D, nous testons des pipelines de perception pour le vol semi-autonome avec évaluation de la latence de réponse de ces pipelines. Le banc d'essai montre ses atouts pour le développement d'expériences homme-drone. Le système distribué ROS gere la distribution des ressources de la plate-forme, et fournit ainsi une abstraction pour se concentrer sur l'optimisation des pipelines. 

La deuxième approche explore le potentiel des systèmes embarqués pour aider au déploiement réel des UAV. Des applications sont explorées pour les drones en tant que plates-formes de détection mobiles, avec un équipement à échantillonnage élevé et de haute précision. Nous concevons deux systèmes embarqués d'acquisition de données et nous les mettons en pratique selon les normes définies par des professionnels du domaine. Une sonde vibratoire a été conçue et nos tests démontrent sa pertinence dans le domaine de la détection mobile. L'utilisation d'un drone porteur s'avère bénéfique pour des opérations de télédétection à haute fréquence et à haute sensibilité. De la détection de pas, aux changements brusques de luminosité, on passe par l'optimisation de l'amortissement d'un drone, à la caractérisation d'un support d'accéléromètre. La collecte de données embarquées est un terrain d'expérimentation intéressant pour l'innovation.