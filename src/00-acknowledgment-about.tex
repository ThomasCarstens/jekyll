\section{Acknowledgements}

% “If you want to go fast, go alone. If you want to go far, go with others”
% – African proverb. There are others that helped this project come together.

% First, Remi Mercier and Daniel Leroy at AllianTech.  My internship was done at home in a first part, and eventually in the design office. In both instances, I want to thank Remi Mercier for being a great source of knowledge and of enthusiasm. The rest of the team welcomed me very warmly, and I trust that AllianTech will be able to make good use of our work.


% This thesis was developed in the laboratory, but also at home, during a time of a global pandemic. I wish to thank my family and my friends from South Africa, who I owe much of this endeavour.

\begin{figure*}[!h]
      \includegraphics[width=4cm, left]{images/dvic.png}
    %   \caption{Examples of mini-UAVs used for remote sensing, from \cite{xiang_xia_zhang_2020}.} 
    %   \label{fig:mini_uavs}
\end{figure*}

\textit{Chapter \ref{c1}. \nameref{c1}} was developed at the De Vinci Innovation Center. Clement Duhart of the DVIC showed much faith to enable such a project.

\textit{Chapter \ref{c2}. \nameref{c2}} was developed with the help of the DVIC Drone Team. Adem, Matt and Berenice, thank you for joining yourselves to me. I trust that future generations of students will make good use of this work. 

The De Vinci Innovation Center is a growing community of students, startups, hackers and fabricators at La Défense, Paris. They have a laissez-faire that lends itself to multi-disciplinary projects and a growing innovation laboratory.                             

\begin{figure*}[!h]
      \includegraphics[width=6cm, left]{images/stage_sota/alliantech_logo.png}
    %   \caption{Examples of mini-UAVs used for remote sensing, from \cite{xiang_xia_zhang_2020}.} 
    %   \label{fig:mini_uavs}
\end{figure*}

\textit{Chapter \ref{c3}. \nameref{c3}} was developed with members of AllianTech, a company specialised in the supply of sensors and sensor system technologies. 

While AllianTech are rooted in France, they liaise with global brands on subjects of acceleration, angular speeds, force, pressure, temperature and mechanical vibrations. Aside from their sales and marketing department, they have a  technical service to adapt products to the client, offer after-sale support and assembly solutions for several clients. Their head office is based in Paris and they have a design office in Toulouse.

A final thanks to Clement Duhart and Marc Teyssier, without whom I would not have strived for this goal. 
% \begin{figure*}[!h]
%   \includegraphics[height=3cm]{images/dvic.png}
%   \includegraphics[height=3cm]{images/stage_sota/alliantech_logo.png}
  
% \end{figure*}

% % “If you want to go fast, go alone. If you want to go far, go with others”
% % – African proverb. There are others that helped this project come together.

% % First, Remi Mercier and Daniel Leroy at AllianTech.  My internship was done at home in a first part, and eventually in the design office. In both instances, I want to thank Remi Mercier for being a great source of knowledge and of enthusiasm. The rest of the team welcomed me very warmly, and I trust that AllianTech will be able to make good use of our work.


% % This thesis was developed in the laboratory, but also at home, during a time of a global pandemic. I wish to thank my family and my friends from South Africa, who I owe much of this endeavour.

% \begin{figure*}[!h]
%       \includegraphics[width=4cm, left]{images/dvic.png}
%     %   \caption{Examples of mini-UAVs used for remote sensing, from \cite{xiang_xia_zhang_2020}.} 
%     %   \label{fig:mini_uavs}
% \end{figure*}

% \textit{Chapter \ref{c1}. \nameref{c1}} was developed at the De Vinci Innovation Center. Clement Duhart has enable projects as ambitious as this one.

% \textit{Chapter \ref{c2}. \nameref{c2}} was developed with the help of the DVIC Drone Team. I trust that future generations of students will make good use of this work. 

% The De Vinci Innovation Center is a growing community of students, startups, hackers and fabricators at La Défense, Paris. They have a laissez-faire that lends itself to multi-disciplinary projects and a growing innovation laboratory.                             

% \begin{figure*}[!h]
%       \includegraphics[width=6cm, left]{images/stage_sota/alliantech_logo.png}
%     %   \caption{Examples of mini-UAVs used for remote sensing, from \cite{xiang_xia_zhang_2020}.} 
%     %   \label{fig:mini_uavs}
% \end{figure*}

% \textit{Chapter \ref{c3}. \nameref{c3}} was developed with members of AllianTech, a company specialised in the supply of sensors and sensor system technologies. 

% While AllianTech are rooted in France, they liaise with global brands on subjects of acceleration, angular speeds, force, pressure, temperature and mechanical vibrations. Aside from their sales and marketing department, they have a  technical service to adapt products to the client, offer after-sale support and assembly solutions for several clients. They currently run a head office in Paris and a design office in Toulouse.

% A final thanks to Clement Duhart and Marc Teyssier without which I wouldn't have strived for this goal. 
% % \begin{figure*}[!h]
% %   \includegraphics[height=3cm]{images/dvic.png}
% %   \includegraphics[height=3cm]{images/stage_sota/alliantech_logo.png}
  
% % \end{figure*}







\newpage
\section{Author Background}

\vspace{3cm}
Thomas Carstens holds a dual Bachelor’s degree in Mechanical and Mechatronic Engineering from the University of Cape Town, South Africa. This thesis is the final step of the Engineering Masters at ESILV, in Paris, France. For the ESILV Masters, Thomas studied the Creative Technologist Track. He documents his latest work on his personal website: {https://thomascarstens.github.io/}

% \textbf{Clément DUHART} is a researcher interested in Ambient Intelligent Systems in which Internet of Things meets the Artificial Intelligence. This is relevant on how we can design systems able to evolve and solve problems by them self in order to improve the management of our environments and to avoid people slavery by technology. Indeed the developpment of information systems are pushing more and more information on our devices which floods our daily. It's time that technology learns how to manage itself and that human recovers an unplugged world without loosing any technology advantage.

% Since last 5 years, he is researcher-teacher and gives conferences and practical courses on how to design systems based on Artificial Intelligence, Embedded Systems, Wireless Sensor Networks to take part in the formation of the engineer generation for the Internet of Things.

% He received two M.Sc. degrees in 2012 from the University of Pierre et Marie Curie Paris VI in Artificial Intelligence and Decisions and from ECE Paris Engineering School in Embedded Systems. He got his Ph.D. degree in Computer Science in Le Havre University in 2016 applied on Organic Ambient Intelligence before to join the Responsive Environment Group at MIT Medialab.
